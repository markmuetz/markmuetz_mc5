\documentclass[11pt,a4paper]{article}

\usepackage{siunitx}
%\usepackage[version=4]{mhchem}
\usepackage{multirow}
\usepackage{subfig}

\usepackage{pgfgantt}
%\usepackage{pdflscape}
% \usepackage[a4paper,margin=1in,landscape]{geometry}

% \usepackage[pdftex]{color,graphicx}
% \pagestyle{plain}
\usepackage{geometry}
\usepackage{rotating}
\usepackage{hyperref}

\newcommand{\ts}{\textsuperscript}
\newcommand{\ic}{\texttt}
\newcommand\todo[1]{\textbf{TODO: #1}}

\sisetup{detect-weight=true, detect-family=true}

\usepackage[backend=biber,style=authoryear,sorting=nyt,dashed=false]{biblatex}
\renewcommand*{\nameyeardelim}{\addcomma\space}
\addbibresource{references/references.bib} % note the .bib is required

%Wrong spellings!
%parameterization (unless part of someone else's work)
%parameterizing
%Paracon

\begin{document}

% \newgeometry{margin=2.0cm}
\newgeometry{margin=2.2cm, top=2.5cm}

\begin{center}
    \Large{\textbf{Monitoring Committee Report V}}\\[0.1cm]
    \large{Mark Muetzelfeldt}\\
    \normalsize{11am on Thursday 14\ts{th} December 2017 in 2U13}\\[0.1cm]		
    \rule{\textwidth}{0.2mm}
    \textbf{Project Title: }Development of scale-awareness in the representation of
    convective cloud systems\\
    \textbf{Monitoring Committee: }Dr Omduth Coceal and  Dr Andrew Turner\\
    \textbf{Supervisors: }Prof. Robert Plant, Prof. Peter Clark, Dr Steve Woolnough \\
    and Dr Alison Stirling (Met Office CASE supervisor)\\
    \rule{\textwidth}{0.2mm}
\end{center}

\section{Project overview}
\label{sec:Project Overview}
% Where have I been, where am I and where am I going.

% Overall goal.
% * represent the org of convection in a conv p13n
% * do in a scale-aware way.
% Steps to do this (see thesis structure notes)
% Stress physically justifiable
% Point out links/deps between hi-res <=> p'trized
With Monitoring Committee V comes the extra requirement of detailed planning of the rest of my PhD. This is a useful exercise in its own right, and focuses my mind on the tasks of generating the results and writing up my thesis. It is also a good time to take stock and think about what I have done to date, how to bring this together into a coherent thesis and what gaps there might be in this thesis.

The first part of my PhD getting up-to-speed with meteorology and numerical modelling, completing 6 assessed modules. This was followed by learning about the fields of convective parametrization and high-resolution modelling, putting into practice a lot of what I learnt by running the idealized UM (iUM). In phase 0 of my PhD plan, the goal was to make sure that the iUM was capable of being used for what we wanted to do; this took longer than expected due to the moisture conservation issues highlighted in previous MCs. Subsequent work, leading up to the companion paper for MC IV, built on learning how to use the iUM and on understanding and managing its flaws for high-resolution modelling. I also spent time reading up on the organization of convection, in the atmosphere in the form of squall lines, and in models in the form of squall lines, and through the mechanism of self-aggregation.

To indicate how I will turn this work into a thesis, it is useful to set out the logical line of argument of my thesis. The goal remains to represent shear-induced organization of convection in a convective parametrization scheme, ideally in a way that is scale-aware. The line of argument starts by recognizing that shear can have an organizing effect on cloud fields. By looking at the range of wind profiles generated in coarse-resolution models, I will build a number of Representative Wind Profiles (RWPs), that reduce the parameter space to something manageable. The second strand is to use high-resolution models to investigate how shear-induced organization affects key aspects of the cloud field, for example the lifetime of the convection, the bulk entrainment, the vertical momentum transports and the statistics of mass flux in the cloud field. This information, coupled with a diagnosis of which RWP a given grid-column is in, can then be used to modify the convective parametrization scheme in the coarse-resolution model.

There are still blanks to be filled in in the remaining part of my PhD. How to generate the RWPs? I have done some experimental work to classify these profiles using a machine learning technique, but this needs to be extended and the analysis checked more thoroughly (see Section \ref{sec:Classification of shear profiles}). How to modify the convective parametrization? We have some ideas how this could work (see Section \ref{sec:um_mod}), but these will need to be tested. 

Overall, I am pleased with how my PhD is going. Although I have not accomplished all the things laid out in last MC report's ``Future work'' section, I believe I have done an equivalent amount of work. My placement at the Met Office (Section \ref{sec:Met Office placement}) meant that it was sensible to re-arrange the order in which I carried out my PhD plan, looking into how I might go about modifying a convective parametrization scheme before carrying out some of the high-resolution modelling. Finding a potential way of reducing the parameter space of the shear profiles also means that the high-resolution modelling has been moved further back in my updated plan. 

\subsection{Background reading}
\label{sec:Background reading}

% Conv p13n + org.
% Mapes and Neale 2011
% Willet and Whitall 2017
% Tompkins and Semie 2017
% Shear climatology
% Aiyyer and Thorncroft 2006
% Houchi et al. 2010
% Bonus
% Davies et al. 2009

\todo{check tense}

To start to get to grips with some work looking at wind profile and shear climatologies, I read \cite{houchi2010comparison}. They look into the zonal and meridional profiles of $u$ and $\frac{du}{dz}$ in both radiosonde observations and ECMWF \SI{12}{hr} forecasts. Although the motivation of the work is different from mine, they are interested in using the ECMWF forecasts to model how a satellite will perform, some of the ideas may be useful and it is something to compare against. However, they were looking at comparisons between co-located observations and interpolated model output, and they were interested in a greater depth of atmosphere, from the surface to \SI{35}{km}.

In \cite{mapes2011parameterizing}, they look at the link between entrainment as formulated in a convective parametrization and the behaviour of the model. They identify the ``entrainment dilemma'', whereby a value of the entrainment rate that is too high leads to a model that is unstable to explicit convection, and conversely, an entrainment rate that is too low leads to too much deep convection, and a corresponding deterioration of the space and time distributions and variability. To ameliorate this situation, they develop a 2D dimensionless prognostic \textit{org} that represents the subgrid organization of a given grid-column. \textit{org} is advected around by mass-weighted mean of the wind field over the convective layer, which defines an effective steering level. Rain evaporation is taken as a source of \textit{org}, which can be advected horizontally. \textit{org} acts to reduce the entrainment rate in a grid-column, thereby acting as a time-lagged positive feedback on the parametrized convection. 

\cite{willett2017simple} take a similar approach. They again change the entrainment rate through a new prognostic variable. This time it is a 3D prognostic variable, which they argue makes it simpler to advect around as there is no need to work out an effective steering level as in \cite{mapes2011parameterizing}. The source of the prognostic is taken as being the precipitation, as this is related to the strength of the convection.

Finally, in \cite{tompkins2017organization} they examine how entrainment affects organization from a different perspective - that of a Cloud Resolving Model (CRM). In my project, I am aiming to link entrainment in the convective parametrization to results from a CRM, so this serves as an introduction to some of the concepts involved. \todo{add more}

\section{Completed work}
\label{sec:Completed work}

\subsection{Progress overview}
\label{sec:Progress overview}

In terms of sticking to the work laid out in the last MC, I have made good progress on cloud tracking (Section \ref{sec:Cloud tracking}). I have also switched to performing the simulations at \SI{1}{km} resolution. To reduce the computation time, I worked on reducing the spin-up time by using lower-resolution profiles of $\theta$ and water vapour, work which will be even more useful in terms of saving computation time at higher resolution. I have not done more work at \SI{250}{m} resolution yet, nor have I run a stochastic parametrization scheme in either idealized or global simulations. Instead, I laid the foundations for other parts of my PhD with the shear classification work and making modifications to the \cite{gregory1990mass} scheme as set out in Sections \ref{sec:Classification of shear profiles} and \ref{sec:um_mod} respectively.

\subsection{Cloud tracking}
\label{sec:Cloud tracking}

%\cite{wilson1998nowcasting},  \cite{plant2009statistical}

From analysing cloud tracks, useful information can be gleaned on e.g. cloud lifetimes that I might be able to use to modify a convective parametrization scheme. The cloud tracking used here uses an approach that is a combination of Juwon Kim's cloud tracking and the method of \cite{plant2009statistical}. 
%The reason that I chose not to use Juwon's work without modification was that it was developed for a different reason - tracking clouds in real world situations with a changing wind vector across the domain (my idealized work used a constant wind over the domain). \cite{plant2009statistical} method was integrated directly into the model, this was not possible for me although the bulk of the ideas come from this method. 
The method works by comparing two snapshots of the cloud field at a given height a certain time apart, in my case I use \SI{2}{km} and \SI{5}{min}. To calculate overlaps, the maximum correlation vector between the two fields is calculated, and this is used to project the cloud field as it was in the first snapshot to where it would have moved to by the time of the second snapshot. This idea and code was taken from Juwon's method. Then the method of \cite{plant2009statistical} is applied, working out the number of merges and splits and handling edge cases such as when two clouds split and merge into two separate clouds (see Figure \todo{XXX}).

I then applied this method to data from the last \todo{XXX} days from a \SI{1}{km} resolution run, the results from the no shear and strong shear experiments (S0 and S4) are presented in Figure \ref{fig:cloud_lifetimes}. 
The distribution is bimodal, with the first mode showing a large number of short-lived clouds. The second mode peaks at a lifetime of around \SI{50}{min}. 
The results from S0 for all cloud lifetimes are in line with previous modelling work \cite{plant2008stochastic}.

The distribution can be further analysed by splitting it into simple and complex lifetimes - lifetimes that contain no merges or splits and those that do. The complex lifetimes distribution for S0 is shown in subfigure \ref{fig:cloud_lifetimes} (b), where it is seen that complex lifetimes are longer than all lifetimes, and there are far fewer with very short lifetimes. 

Comparing the equivalent subfigures for S4, two main points jump out. The first is that the number of short-lived clouds for all lifetimes is increased. Speculatively, this could be due to enhanced shear above the boundary layer increasing entrainment and therefore decreasing cloud lifetime, and so leading to a large number of short-lived clouds. The second is that for S4, the distribution is shifted towards longer lifetimes. The mean lifetime for these clouds is \todo{XXX}, whereas for S0 it is \todo{XXX}.

\begin{figure}[htp!]%
    \centering
    \subfloat[S0, all lifetimes]{{\includegraphics[width=7cm]{figs/atmos_S0_1km_cloud_tracking_z1_t1_all_lifetimes.png} }}%
    \qquad
    \subfloat[S0, complex lifetimes]{{\includegraphics[width=7cm]{figs/atmos_S0_1km_cloud_tracking_z1_t1_nonlinear_lifetimes.png} }}%
    \qquad
    \subfloat[S4, all lifetimes]{{\includegraphics[width=7cm]{figs/atmos_S4_1km_cloud_tracking_z1_t1_all_lifetimes.png} }}%
    \qquad
    \subfloat[S4, complex lifetimes]{{\includegraphics[width=7cm]{figs/atmos_S4_1km_cloud_tracking_z1_t1_nonlinear_lifetimes.png} }}%
    \caption{Distribution of cloud lifetimes for S0 (top) and S4 (bottom). Left shows all cloud lifetimes, right shows complex lifetimes only. Statistics were gathered over 10 days of simulation.}%
    \label{fig:cloud_lifetimes}%
\end{figure}

\subsection{Classification of shear profiles}
\label{sec:Classification of shear profiles}

My work so far has been oriented towards understanding how shear affects cloud fields. However, the aim is to information from this work to modify a convective parametrization scheme. This requires being able to link the shear as seen by a model grid-column to some change to the parametrization scheme. One way of doing this is to classify the shear in a grid-column, as presented here.

To find the shear profiles for classification, I ran a standard Met Office GA7.0 run (vn10.6), at N96 resolution. I ran this for one day, and took results from the tropics (defined as \ang{30}N - \ang{30}S) at a frequency of once every \SI{6}{hr}. I outputed the $u$ and $v$ profiles on 6 pressure levels of: \SI{925}, \SI{850}, \SI{700}, \SI{500}, \SI{300} and \SI{200}{hPa}. The profiles were optionally filtered based on whether there was convective instability (CAPE $>$ \SI{500}{J.kg^{-1}}) or grid-scale ascent (w $>$ \SI{0}{m.s^{-1}}). To group the profiles, I turned each pair of $u$ and $v$ profiles into one sample with 12 values or dimensions, and ran a Principal Component Analysis algorithm on it to reduce the number of dimensions to 5 (this explains over 90\% of the variance of the profiles). I then ran a k-means clustering algorithm to group the profiles, choosing to produce 20 individual clustered groups. The algorithms were taken from scikit-learn \parencite{scikit-learn}.

Figure \ref{fig:shear_profiles} shows the results from this analysis, with the profiles being filtered on CAPE $>$ \SI{500}{J.kg^{-1}}. Subfigure (a) shows a reasonably successful grouping, the low level shear is aligned for both wind components and the spread between the profiles is not too large. Subfigure (b) shows a group of profiles where the process has been less successful; there is large spread between the different profiles at all levels and important features in some profiles are not seen in others.

This is a work in progress. I would like to continue this over the next couple of months because I believe it holds promise for linking the two strands of my thesis. My supervisors and I have discussed ways in which this could be improved, such as deciding which of these profiles are interesting from a shear perspective, processing the profiles first to normalize them about a given shear angle and removing the dependence on higher levels as this is unlikely to affect shear induced organization.

\begin{figure}[htp!]%
    \centering
    \subfloat[]{{\includegraphics[width=7cm]{figs/atmos_None_shear_profile_classification_analysis_use_pca-True_filt-cape_profile-5_n_clust-19_ci-12.png} }}%
    \qquad
    \subfloat[]{{\includegraphics[width=7cm]{figs/atmos_None_shear_profile_classification_analysis_use_pca-True_filt-cape_profile-5_n_clust-19_ci-17.png} }}%
    \caption{\todo{invert y-axis!} Shear profiles classified into different RWPs, $u$ is blue and $v$ is red. (a) shows reasonable classification success with relatively distinct profiles classified together, whereas (b) shows large variation between the individual profiles.}%
    \label{fig:shear_profiles}%
\end{figure}


\subsection{Modifications to the UM based on shear}
\label{sec:um_mod}

When I was on my Met Office placement, I learnt how to make modifications to the UM. The goal was to add a link between the shear in a grid-column to the convection scheme. The emphasis at this stage was more in terms of the technicalities of how to make the changes rather than exactly what changes to make, although the final changes I make may well end up being similar to what I have done. The change was to modify the entrainment rate of the scheme based on the shear in a grid-column, taking a cue from my work looking at cloud lifetimes in CRMs and arguing that reduced entrainment is likely to increase the lifetime of a cloud. The difficulty is that the work I have done on lifetimes suggests that I should change something corresponding to the cumulus lifetime in the parametrization scheme, however the UM convection scheme \parencite{gregory1990mass} has no such parameter. Therefore entrainment was chosen as an expedient proxy for lifetime for the time being. N.B. other schemes, such as \cite{plant2008stochastic}, do represent cloud lifetime explicitly, and so might be more amenable to this modification. 

The modifications involved adding in new diagnostics: the bulk shear and the Bulk Richardson Number (BRN). They also involved passing namelist settings into the UM to e.g. turn the new `shear aware' scheme on. The shear information had to be passed into the convection scheme to where the entrainment rate is calculated, and used to modify this. Given that my knowledge of Fortran had been fairly superficial before now, I also had to learn how to programme in it and about some of its quirks. I made changes in keeping with the Met Office's best practices, creating a ticket (#3599) and branch to describe and implement that changes.

To verify the new diagnostics I had added were correct, I outputted some standard ($\theta$, $u$, $v$, orography) to perform the same calculation offline in Python. This involved careful matching up of horizontal C-grids, Charney-Phillips staggered vertical grids and $\sigma$ scaled height coords as well as outputting the standard data one timestep before the new diagnostic. After numerous bugs, mainly due to array indexing/passing into subroutines, had been found and eliminated, good agreement was seen between the two calculations (differences were small and I believe due to the packing precision of the diagnostics).

From here I could use the BRN to affect the deep convection scheme. The change was rather arbitrary - a BRN $<$ 10 was used to reduce the entrainment by 5\%. Examined over the course of 1 day, the change had a small but growing effect on the model state, indicating that the code was being exercised and was having an effect. I will be able to build on this work when it comes to running the parametrized tests later in my PhD.

\section{Future work}
\label{sec:Future work}

An updated timetable for my future work is set out in a Gantt chart in the appendix (Figure \ref{gantt}). My aim is to have finished the high-resolution modelling by the start of April 2018, in time for EGU (8\ts{th} - 13\ts{th} April). I will also have a paper written up on this work by the next MC (around end of June 2018).

\subsection{Shear classification and climatology in the UM}
\label{sec:Shear climatology in the UM}

Continuing the work I have done with the classification of shear profiles, I will extend it to make a shear climatology. This will serve as a basis for modifying the UM's convection scheme based on the shear, by getting a handle on where and  when the various shear profiles are present and also for how to detect and classify these profiles. It will include some experimentation with the best way of performing the classification, and is therefore subject to perhaps the most uncertainly in my future plans. The remaining sections also depend on this, so it seems like a good idea to tackle this first. One of the results from this will be a set of RWPs, which will be useful for subsequent work. However, if it does not work as expected, or the profiles are not usable because e.g. the groups produced are not representative enough, I have the fallback option of using the work that I have already done with linearly scaled shear profiles along with arguments about how to diagnose similar states in the model based on BRN.

\subsection{High-resolution idealized modelling}
\label{sec:High-resolution idealized modelling}

The first part of the work be a repeat of what I did for MC IV's companion paper, using the most up-to-date version of the idealized UM with Adrian Lock's `optimum' settings, which are meant to have improved moisture conservation. In the first section, I will run simulations at \SI{2}{km} (as before), \SI{1}{km} and \SI{250}{m} resolution with linearly scaled shear profiles based on the profile that is known to produce organization. The analysis will include everything in the previous companion paper, as well as a section of cloud lifetime statistics, improved analysis of the thermodynamic state, and outputting of the diagnosed momentum transport by convection. It will hopefully include analysis of the entrainment in the model, a more versatile metric for organization and a mechanism analysis of what causes the increased lifetime (through separation of the updraught and downdraught).

Ideally, I will also use the RWPs from the previous section. The RWPs can be used to drive idealized simulations in the same way that the linearly scaled shear profiles can. The above measures can all be used with these profiles to generate behavioural responses to each of the RWPs, that can then be used in the convection scheme along with the classification into RWPs of a given grid-column profile.

\subsection{Writing}
\label{sec:Writing}

I will write up the work I have done on shear climatologies in the UM and high-resolution modelling as a paper. This will include all the analysis from the previous two sections. It will be complete or very close to complete by the next MC, and will therefore be a good indicator of progress at that I am sticking to the plan.

\section{Training record}
\label{sec:Training record}

% CMSS
% RRDPs
I took part in the biennial NCAS Climate Modelling Summer School, in Cambridge from the 11\ts{th} to the 15\ts{th} of September. I helped out in Hilary Weller's Introduction to Numerical Modelling classes, which were broadly similar to her MSc classes although condensed down from one term to one week. There were a wide range of abilities, from people who had never used Python to people doing PhDs on numerical techniques and people with a strong grasp of Python. This kept me on my toes and was a great experience of a fast-paced learning environment. I took part in judging the poster competition, and now appreciate how difficult it is to try and see 30 posters in an hour and a half. I also attended several of the lectures, so was able to learn a bit about how e.g. ocean modelling is done.

I completed 2 of my 3 RRDPs, going to courses on ``Open access research research publication'' in Reading, learning what the requirements are for me as a NERC funded student. I also went to an ``Introduction to impact'', learning what the definition of impact is and what is required in a ``Pathways to Impact'' assessment. 

\subsection{Met Office placement}
\label{sec:Met Office placement}

I spent most of November at the Met Office, working with convective parametrization group, part of Atmospheric Processes and Parametrization. I was working on making modifications to the UM, as set out in Section \ref{sec:um_mod}. I discussed my plans for the modifications with Alison Stirling, my Met Office CASE supervisor, getting many good ideas from her with regards to how to modify the scheme, and how to extend my high-resolution modelling work to facilitate this. I talked about the modifications to the scheme with the wider parametrization group, discussing the problem of the fact that \cite{gregory1990mass} does not have a parameter for cloud lifetime and what I could do about this. And I talked about the process involved with actually making changes to the scheme, getting a lot of good input from Rachel Stratton on how to go about doing this. I left with a stack of new papers to read and some new ideas to try out, and having made connections at the Met Office. I also have the beginnings of a branch to the UM code that makes the convective scheme `shear aware', and so the time I spent there was beneficial in many ways.

\subsection{Posters, presentations and conferences}
\label{sec:presentations}
% Cambridge plenary
I went to the Cambridge ParaCon plenary in July, \todo{what to say?}.

% Delft
I attended an international conference in Delft: ``The Future of Cumulus Parametrization'', presenting a poster and attending all of the seminars. \todo{refer to notes from Delft}. 

\todo{Planning to go to EGU}

\subsection{Transferable skills}
\label{sec:Transferable skills}
% PGR forum co-chair + org of QV.
As co-chair of the PGR forum, I was responsible for welcoming the new intake of PhD students into the PGR, helping them find their feet and find their PhD buddies during welcome week. I gave a short course on the University IT systems, and gave a short presentation on my route to becoming a PhD student. As a co-chair, I will be part responsible for the organizing and running of \textit{Quo Vadis} in February next year.

\printbibliography[title={References}]

\newpage
\section*{Appendix}

\renewcommand\thefigure{A.\arabic{figure}}
\setcounter{figure}{0}    
\subsection*{PhD Timetable}

\begin{figure}[htp!]
    \begin{ganttchart}[vgrid, hgrid, y unit chart=0.75cm, MC/.style={milestone/.append style={shape=circle}}]{1}{20}  % <---
        \gantttitle{2017}{2}
        \gantttitle{2018}{12}
        \gantttitle{2019}{6} \\
        \gantttitle{N}{1}
        \gantttitle{D}{1}
        \gantttitle{J}{1}
        \gantttitle{F}{1}
        \gantttitle{M}{1}
        \gantttitle{A}{1}
        \gantttitle{M}{1}
        \gantttitle{J}{1}
        \gantttitle{J}{1}
        \gantttitle{A}{1}
        \gantttitle{S}{1}
        \gantttitle{O}{1}
        \gantttitle{N}{1}
        \gantttitle{D}{1}
        \gantttitle{J}{1}
        \gantttitle{F}{1}
        \gantttitle{M}{1}
        \gantttitle{A}{1}
        \gantttitle{M}{1}
        \gantttitle{J}{1} \\
        \ganttmilestone[MC, milestone left shift=0.2,milestone right shift=-0.2]{Monitoring committees}{2} 
        \ganttmilestone[MC, milestone left shift=0.2,milestone right shift=-0.2]{}{8} 
        \ganttmilestone[MC, milestone left shift=0.2,milestone right shift=-0.2]{}{14}  \\
        \ganttbar{Shear climatology}{2}{3} \\
        \ganttbar{High-resolution modelling}{3}{5} 
        \ganttmilestone{}{5} \\
        \ganttbar{Paper writing}{4}{8} 
        \ganttmilestone{}{8} 
        \ganttbar{}{12}{15} \\
        \ganttbar{Idealized parametrized}{9}{11} \\
        \ganttbar{Global parametrized}{11}{13} 
        \ganttmilestone{}{13} \\
        \ganttbar{Thesis writing}{6}{17} 
        \ganttmilestone{}{17} 
    \end{ganttchart}
    \caption{Gantt chart showing remaining timetable and tasks. Milestones shown as diamonds. The high-resolution modelling finished by EGU (start of April 2018), and the remaining milestones are near to MCs, allowing for progress checks at MC VI and MC VII. Thesis writing will ramp up starting in Q2 next year (or possibly slightly before). The finish date is 1\ts{st} April, 2019.}
    \label{gantt}
\end{figure}

%\subsection*{Repositories}
%
%\begin{itemize}
%  \item managing UM output: \href{https://github.com/markmuetz/omnium}{omnium}
%  \item high-resolution analysis: \href{https://github.com/markmuetz/scaffold_analysis}{scaffold\_analysis}
%  \item climatology of shear analysis: \href{https://github.com/markmuetz/cosar_analysis}{cosar\_analysis}
%\end{itemize}

\subsection*{Training record}
\subsubsection*{Year 1}

\begin{itemize}
  \item RRDP: Intermediate/Advanced \LaTeX\ (4/11/2015)
  \item RRDP: You and your supervisor (11/11/2015)
  \item RRDP: Quality assurance in research (18/11/2015)
  \item RRDP (equivalent): UM Training (16-18/12/2015)
  \item RRDP (equivalent): Preparing to teach: Introduction to teaching and learning (26/1/2016)
  \item Preparing to teach: Marking and feedback (26/1/2016)
  \item Preparing to teach: Laboratory demonstrating and leading small groups (27/1/2016)
  \item MONC Training course (9-10/2/2016)
  \item RRDP (equivalent): Fairbrother Lecture ``A slippery situation: melting ice in Antarctica'' (4/5/2016)
  \item ECMWF Parametrization of subgrid physical processes (16-20/5/2016)
\end{itemize}

\subsubsection*{Year 2}

\begin{itemize}
  \item RRDP: Managing your research project (17/11/2016)
  \item RRDP: How to write a thesis (24/1/2017)
  \item SCENARIO Data Assimilation Course (14-15/2/2017)
  \item RRDP: Presentation skills (7/3/2017)
  \item Software Development for scientists (8/3/2017, 28-29/3/2017)
\end{itemize}

\subsubsection*{Year 3}

\begin{itemize}
  \item NCAS Climate Modelling Summer School: demonstrating Numerical Methods for Hilary Weller (11-15/11/2017)
  \item CASE Met Office Placement (30/10/2017 - 24/11/2017)
  \item RRDP: Open access for research publications (27/11/2017)
  \item RRDP: Introduction to impact (30/11/2017)
\end{itemize}

\subsection*{Talks and conferences attended}

\begin{itemize}
  \item Climate Change 2013: The physical science basis. Institute of Physics (2/2014)
  \item Dame Julia Slingo: Taking the planet into uncharted territory: What climate models can tell us about the future (9/2014)
  \item SCENARIO NERC DTP Conference (9/6/2015)
  \item Climate Change in the run-up to the Paris conference: what has Physics got to say? (6/11/2015)
  \item RMetS talk: The risk and vulnerability of Europe to severe convective storms (6/4/2016)
  \item ParaCon Plenary 1 in Reading (27-28/6/2016)
  \item RMetS debate: What will make the public and politicians take climate change more seriously? (5/10/2016)
  \item RMetS talks: Come Rain or Come Shine (19/10/2016)
  \item COP22 Marrakech: Remote participation (11/11/2016)
  \item ParaCon plenary 2 in Leeds (6-7/12/2016)
  \item RMetS talks: Chaos and Confidence in Weather Forecasting (14/12/2016)
  \item ParaCon plenary 3 in Cambridge (3-4/7/2017)
  \item The Future of Cumulus Parametrization, Delft University of Technology (10-14/7/2017)
  \item ParaCon plenary 4 in Exeter (18-19/12/2017)
  \item (Planned) EGU: Vienna (8-13/4/2018)
\end{itemize}

\subsection*{Talks and conferences presented at}

\begin{itemize}
  \item Presentation: ``Effects of Shear on Cloud Field Organization''. \textit{Quo Vadis}, University of Reading (1/2/2017)
  \item Poster: ``Effects of Shear on Cloud Field Organization''. Met Office Academic Partnership (MOAP), Met Office, Exeter (22/2/2017)
  \item Poster: ``Effects of Vertical Shear on Cloud Field Organization and Variability''. The Future of Cumulus Parametrization, Delft University of Technology (10-14/7/2017)
  \item Poster: ``Effects of Vertical Shear on Cloud Field Organization and Variability''. PhD Poster Session (21/9/2017)
\end{itemize}

\end{document}
